\documentclass[12pt, a4paper]{article}

\usepackage[czech]{babel}
\usepackage{lmodern}
\usepackage[utf8]{inputenc}
\usepackage[T1]{fontenc}
\usepackage{graphicx}
\usepackage{amsmath}
\usepackage[hidelinks,unicode]{hyperref}
\usepackage{float}
\usepackage{listings}
\usepackage{tikz}
\usepackage[final]{pdfpages}
\usetikzlibrary{shapes,positioning,matrix,arrows}

\newcommand{\img}[1]{(viz obr. \ref{#1})}

\lstset{basicstyle=\ttfamily,
showstringspaces=false,
commentstyle=\color{red},
keywordstyle=\color{blue}
}


\begin{document}
    \begin{titlepage}

        \centering

        \vspace*{\baselineskip}

        \begin{figure}[H]
            \centering
            \includegraphics[width=7cm]{fav-logo.png}
        \end{figure}

        \vspace*{1\baselineskip}
        {\sc Semestrální práce z předmětu KIV/UPS}
        \vspace*{1\baselineskip}

        \vspace{0.75\baselineskip}

        {\LARGE\sc Realtime online multiplayer hra - Asteroidy \\}

        \vspace{4\baselineskip}

        {\sc\Large Patrik Janoušek \\}

        \vspace{0.5\baselineskip}

        {A17B0231P}\\
        {janopa@students.zcu.cz}

        \vfill

        {\sc Západočeská univerzita v Plzni\\
        Fakulta aplikovaných věd}


    \end{titlepage}


    \tableofcontents
    \pagebreak

    \newpage

    \section{Úvod}
    \section{Protokol}
    \subsection{Specifikace}
    Protokol byl navržen s ohledem na jednoduchost implementace, ale aby zároveň splnil všechny požadavky, které by aplikace mohla mít.
    
    Mezi hlavní požadavky na protokol byla identifikace typu zprávy.
    Tu je možné provádět pomocí trojciferných číselných identifikátorů.
    Díky těmto identifikátorů je schopná komunikující protistrana předpokládat strukturu příchozích dat.

    Podporuje tedy identifikaci zpráv, kde si klient může určit vlastní identifikátor zprávy, a server tento identifikátor zapíše i do odpovědi, díky čemuž je klient schopen poznat na jakou zprávu server odpovídá.
    Tento identifikátor má variabilní délku, tudíž je zde i možnost jeho nepoužití.

    Tělo zprávy je pak zakódované ve formátu JSON, kde je pro klíče použita konvence snake case. 

    \begin{table}[H]
        \centering
        \begin{tabular}{|l|c|c|}
            \hline
            Popis & Délka & Hodnota\\
            \hline
            \hline
            Dělící znak & 1 & \uv{|} \\
            \hline
            Typ zprávy & 3 & Číslo v ASCII \\
            \hline
            Dělící znak & 1 & \uv{|} \\
            \hline
            Délka těla zprávy & variabilní & Číslo v ASCII \\
            \hline
            Dělící znak & 1 & \uv{|} \\
            \hline
            Identifikátor požadavku & variabilní & Řetězec znaků a-z, A-Z a 0-9 v ASCII \\
            \hline
            Dělící znak & 1 & \uv{|} \\
            \hline
            Tělo zprávy & variabilní & Data zakódovaná v JSONu \\
            \hline
        \end{tabular}
        \caption{Specifikace protokolu}
    \end{table}

    \section{Implementace}
    \subsection{Popis hry}

    \subsection{Definice použitých zpráv}
    Jak již bylo uvedeno výše, každý typ zprávy má svůj unikátní trojciferný identifikátor.
    Tento identifikátor může být z hlediska protokolu libovolný, nicméně pro lepší orientaci byla v konkrétní implementaci vytvořena určitá pravidla.
    
    A to taková, že identifikátor 1xx je použit pro obecné zprávy, které nemusí přímo souviset se samotnou aplikací (keep-alive, atd\dots).
    2xx pak slouží pro zprávy zasílané klientem, a 3xx zprávy zasínalé serverem.
    Zde je ještě dodržováno to, že \uv{xx} je stejné pro klientskou i serverovou zprávu, pokud je serverová zpráva odpovědí na klientský požadavek.
    Tyto identifikátory jsou použity pro zprávy, které přímo nesouvisí se hrou, ale spíše s její správou (autentizace, založení lobby, získání seznamu hráčů, \dots).
    Nakonec jsou pak použity identifikátory zpráv 4xx a 5xx, kde 4xx jsou zprávy zasílané klientem, a 5xx zprávy zasílané serverem. Tyto identifikátory jsou použity pro zprávy přímo související se hrou (pohyb hráče po herní ploše, střelba, \dots).

    Všechny zprávy zasílané serverem, jsou obalovány do struktury, která umožňuje k zasílaným datům přidat status, a případně nějakou dodatečnou zprávu. Přenášená zpráva je pak dostupná pod klíčem \uv{data}.

    \begin{table}[H]
        \centering
        \begin{tabular}{|l|c|}
            \hline
            Klíč & Datový typ\\
            \hline
            \hline
            data & object \\
            \hline
            message & string \\
            \hline
            status & boolean \\
            \hline
        \end{tabular}
        \caption{Obalovací zpráva serveru}
    \end{table}

    \subsubsection{Zprávy zasílané klientem}
    \subsubsection*{KeepAlive}
    Typ zprávy: 100\\\\
    Tato zpráva je zasílána klientem, aby ověřila funkčnost spojení se serverem, a zároveň schopnost komunikace mezi těmito dvěmi stranamy.

    \begin{table}[H]
        \centering
        \begin{tabular}{|l|c|}
            \hline
            Klíč & Datový typ\\
            \hline
            \hline
            ping & string \\
            \hline
        \end{tabular}
        \caption{Specifikace KeepAlive}
    \end{table}

    Klient jako hodnotu pole \uv{ping} vyplňuje řetězec \uv{pong}. A server na tuto zprávu odpovídá stejným typem zprávy, jen jako hodnotu pole \uv{ping} vyplňuje řetězec \uv{ping-pong}.

    \subsubsection*{ActionError}
    Typ zprávy: 101\\\\
    Zpráva, která je zasílána serverem jako odpověď na typ zprávy, ke které klient nemá oprávnění, nebo daná zpráva neexistuje.
    Tato zprávy má prázdné tělo a neposílá žádná dodatečná data.

    \subsubsection*{Authenticate}
    Typ zprávy: 200\\\\
    Autentizační zpráva musí být klientem zaslána jako první zpráva po připojení k serveru.
    V případě neposlání této zprávy nemá klient žádné pravomoce, mimo zasílání \textit{KeepAlive}.

    \begin{table}[H]
        \centering
        \begin{tabular}{|l|c|c|}
            \hline
            Klíč & Datový typ & Popis\\
            \hline
            \hline
            name & string & Jméno hráče\\
            \hline
        \end{tabular}
        \caption{Specifikace Authenticate}
    \end{table}

    \subsubsection*{CreateLobby}
    Typ zprávy: 201\\\\
    Zpráva určená k založení lobby.
    Hráč, který lobby vytváří, je automaticky do lobby připojen a není již potřeba posílat \textit{JoinLobby}.

    \begin{table}[H]
        \centering
        \begin{tabular}{|l|c|c|}
            \hline
            Klíč & Datový typ & Popis\\
            \hline
            \hline
            name & string & Název lobby\\
            \hline
            players\_limit & integer & Limit počtu hráčů v lobby\\
            \hline
        \end{tabular}
        \caption{Specifikace Authenticate}
    \end{table}

    \subsubsection*{DeleteLobby}
    Typ zprávy: 202\\\\
    Zpráva určená ke smazání lobby.
    Ačkoliv zpráva přjímá jako parametr název lobby ke smazání, tak hráč má právo smazat lobby jenom za předpokladu, že je jejím vlastníkem.

    \begin{table}[H]
        \centering
        \begin{tabular}{|l|c|c|}
            \hline
            Klíč & Datový typ & Popis\\
            \hline
            \hline
            name & string & Název lobby ke smazání\\
            \hline
        \end{tabular}
        \caption{Specifikace DeleteLobby}
    \end{table}

    \subsubsection*{ListLobbies}
    Typ zprávy: 203\\\\
    Prázdná zpráva sloužící k získání seznamu všech dostupných lobby.

    \subsubsection*{JoinLobby}
    Typ zprávy: 204\\\\
    Zpráva sloužící k připojení do lobby.
    Připojení je možné pouze za předpokladu, že je v lobby volný slot pro hráče.
    Tato skutečnost je ověřována serverem, který vyplní patřičný status zprávy.

    \begin{table}[H]
        \centering
        \begin{tabular}{|l|c|c|}
            \hline
            Klíč & Datový typ & Popis\\
            \hline
            \hline
            name & string & Název lobby k připojení\\
            \hline
        \end{tabular}
        \caption{Specifikace JoinLobby}
    \end{table} 

    \subsubsection*{LeaveLobby}
    Typ zprávy: 214\\\\
    Prázdná zpráva, kterou se hráč odpojí z lobby.

    \subsubsection*{ListLobbyPlayers}
    Typ zprávy: 206\\\\
    Prázdná zpráva sloužící k získání seznamu hráčů přípojených do lobby.

    \subsubsection*{StartGame}
    Typ zprávy: 207\\\\
    Prázdná zpráva pro spuštění hry.
    Tuto zprávu může poslat libovolný hráč z lobby.
    Od této chvíle je na serveru vytvořená instance hry, a všichni hráči jsou neustále informováni o jejím stavu.

    \subsubsection*{GameReconnectAvailable}
    Typ zprávy: 211\\\\
    Touto prázdnou zprávou si může klient ověřit, zda připojený hráč nepatří do nějaké hry, a tudíž se do ní může opět připojit, nebo se z ní odpojit.

    \subsubsection*{Reconnect}
    Typ zprávy: 212\\\\
    Zasláním této prázdné zprávy se hráč připojí zpět do probíhající hry, kterou předtím opustil.

    \subsubsection*{LeaveGame}
    Typ zprávy: 213\\\\
    Zasláním této prázdné zprávy hráč definitivně opustí probíhající hru, ze které odešel, a již se do ní nemůže vrátit.
   
    \subsubsection*{PlayerMove}
    Typ zprávy: 400\\\\
    Zpráva, která informuje server o změně atributů hráče.
    Tuto zprávu by měl klient posílat při změně své pozice, nebo libovolného atributu svého pohybu (vektor pohybu, rotace, \dots).
    Zaslané údaje jsou na serveru verifikovány a v případě chybně zaslaných údajů je klient vyžádán k synchronizaci údajů se serverem (viz. \textit{UpdateState}).

    \begin{table}[H]
        \centering
        \begin{tabular}{|l|c|c|}
            \hline
            Klíč & Datový typ & Popis\\
            \hline
            \hline
            pos\_x & double & Pozice na ose X\\
            \hline
            pos\_y & double & Pozice na ose Y\\
            \hline
            velocity\_x & double & Složka X vektoru pohybu\\
            \hline
            velocity\_y & double & Složka Y vektoru pohybu\\
            \hline
            rotation & double & Rotace hráče v radiánech\\
            \hline
        \end{tabular}
        \caption{Specifikace PlayerMove}
    \end{table}

    \subsubsection*{ShootProjectile}
    Typ zprávy: 401\\\\
    Prázdná zpráva informující server o vystřelení projektilu

    \subsubsection{Zprávy zasílané serverem}
    \subsubsection*{CreatedLobbyResponse}
    Typ zprávy: 301\\\\
    Prázdná zpráva, kterou je klient informován o založení lobby.

    \subsubsection*{DeleteLobbyResponse}
    Typ zprávy: 302\\\\
    Prázdná zpráva, kterou je klient informován o smazání lobby.

    \subsubsection*{ListLobbiesResponse}
    Typ zprávy: 303\\\\
    Tuto zprávu server používá pro zaslání seznamu všech dostupných lobby.
    V kořenu zprávy je jedinný klíč \uv{lobbies}, který má následující strukturu:

    \begin{table}[H]
        \centering
        \begin{tabular}{|l|c|c|}
            \hline
            Klíč & Datový typ & Popis\\
            \hline
            \hline
            name & string & Název lobby\\
            \hline
            connected\_players & integer & Počet hráčů připojených do lobby\\
            \hline
            players\_limit & integer & Maximální počet připojených hráčů\\
            \hline
        \end{tabular}
        \caption{Specifikace ListLobbiesResponse}
    \end{table}

    \subsubsection*{JoinLobbyResponse}
    Typ zprávy: 304\\\\
    Prázdná zpráva informující klienta o stavu připojení do lobby.

    \subsubsection*{PlayerLobbyJoined}
    Typ zprávy: 305\\\\
    Zpráva informující hráče, kteří jsou již připojení v lobby, o připojení nového hráče.

    \begin{table}[H]
        \centering
        \begin{tabular}{|l|c|c|}
            \hline
            Klíč & Datový typ & Popis\\
            \hline
            \hline
            name & string & Jméno hráče\\
            \hline
        \end{tabular}
        \caption{Specifikace ListLobbiesResponse}
    \end{table}

    \subsubsection*{ListLobbyPlayersResponse}
    Typ zprávy: 306\\\\
    Zpráva, kterou server klientovi vrací seznam hráčů připojených do lobby.

    \begin{table}[H]
        \centering
        \begin{tabular}{|l|c|c|}
            \hline
            Klíč & Datový typ & Popis\\
            \hline
            \hline
            players & string[] & Pole jmen hráčů\\
            \hline
        \end{tabular}
        \caption{Specifikace ListLobbyPlayersResponse}
    \end{table}

    \subsubsection*{StartGameResponse}
    Typ zprávy: 307\\\\
    Prázdná zpráva, která se zasílá všem hráčům při startu hry.

    \subsubsection*{LobbyPlayerConnected}
    Typ zprávy: 308\\\\
    Zpráva informující ostatní hráče v lobby o připojení nového hráče.

    \begin{table}[H]
        \centering
        \begin{tabular}{|l|c|c|}
            \hline
            Klíč & Datový typ & Popis\\
            \hline
            \hline
            name & string & Jméno příchozího hráče\\
            \hline
        \end{tabular}
        \caption{Specifikace LobbyPlayerConnected}
    \end{table}

    \subsubsection*{LobbyPlayerDisconnected}
    Typ zprávy: 309\\\\
    Zpráva informující ostatní hráče v lobby o odpojení hráče.

    \begin{table}[H]
        \centering
        \begin{tabular}{|l|c|c|}
            \hline
            Klíč & Datový typ & Popis\\
            \hline
            \hline
            name & string & Jméno odpojeného hráče\\
            \hline
        \end{tabular}
        \caption{Specifikace LobbyPlayerConnected}
    \end{table}

    TODO: 31x

    \subsection{Server}
    Pro implementaci serveru byl zvolen jazyk go.
    Jednak z důvodu vyzkoušení nové technologie, ale i z důvodu jeho technologických výhod.
    Zejména z důvodu jednoduchosti paralelizace a možnosti jednoduché a bezpečné komunikace mezi jednotlivými vlákny (resp. gorutinami), ale i automatické správy paměti.

    \subsubsection{Síť}
    Síť je v serveru implementována s důrazem na co největší oddělení od zbytku aplikace, a je vedle aplikace v samostatném balíku \uv{net}, který je pojmenován po vzoru stejnojmenného balíku ze standardní knihovny jazyka go.
    Tento balík obsahuje implementaci jak samotného TCP serveru, tak i protokolu.
    
    Protokol je plně oddělen od TCP serveru, se kterým komunikuje pomocí rour.
    Díky tomu je protokol schopen komunikovat s libovolným binárním streamem dat, který nutně nemusí přijít od TCP serveru.
    Tato vlastnost ho dělá i samostatně testovatelným.
    Nevýhodou je, že je díky tomu komunikace s protokolem blokující, a každý klient musí mít pro protokol vlastní gorutinu.

    Pokud tedy přijde zpráva, tak jí TCP server přečte ze socketu, a zapíše jí do patřičné roury připojeného klienta, odkud zprávu začne přečte protokol, a zečne jí zpracovávat.
    Po tom, co protokol přečte celou zprávu, tak ji naplní do struktury \texttt{ProtoMessage}, a pomocí kanálu ji předá zpět TCP serveru, který tuto zprávu vezme, přidá k ní odesílatele, a skrz další kanál jí předá master serveru, který může zprávu na základě jejího obsahu a typu dále zpracovávat.

    \subsection{Reakce na zprávy}

\end{document}
